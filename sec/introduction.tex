\section{Introduction}

\vspace{-1.5pt}

Many people in the workforce are familiar with taking drug tests as part of the hiring process. These tests are included to dissuade employees from abusing drugs, as drug abuse has been clinically shown to have a negative impact on productivity.\hspace{1pt}In addition, drug tests are used to prevent hiring those who use illicit drugs, identify and combat early signs of drug abuse, and provide a safe work space for employees\cite{workplace-drug-testing}.While drug tests are a great tool for maintaining employee quality, they can also be somewhat expensive with most drug tests ranging from \$10 to \$30\cite{drug-testing}. For large companies that regularly test their future and current employees, these prices can quickly add up, and with a small number of working adults who are characterized as illicit drug abusers, unnecessary resources are expended on drug testing employees who do not fall into the same category.\\

The goal of this project is to predict whether or not a person may be using illicit drugs through a personality test.\hspace{1pt}In addition to a few basic questions, such as age and education, a respondent would be asked to take a personality test, in which the responses would be passed into our model to output the likelihood of that person being an illicit drug abuser. In the event that our model predicts a respondent to be an illicit drug abuser, a company can request that person to take a drug test, while excluding those who had an opposite prediction. By having the ability to distinguish likely illicit drug abusers from others, a company can save money by not having to expend additional resources in unnecessary drug tests.